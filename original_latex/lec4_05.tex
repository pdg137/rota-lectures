{\Large 18.313 Lecture 20, Monday, April 5, 1999}\newline 
{\large Lecture by Prof. Gian-Carlo Rota}\newline 
Transcribed by Vladislav Gabovich (vyg@mit.edu) and Bilge Demirk\"oz (bilge@mit.edu)\newline

\noindent {\bf Continuous Conditional Probability} (continued):

Our objectives are:
\begin{enumerate}
\item{$P(X_1 \leq s | Y=t)=?$}
\item{$dens(X_1 =s | Y=t)=?$}
\end{enumerate}
Note that the $2^{nd}$ expression is the derivative of the $1^{st}$ and so
the real problem is to find the $1^{st}$ expression.

Lets take an example of a Dirichlet process. 
\begin{equation}
P(X_1 >s | X_{(k)}=t)= ?
\end{equation}
What we can compute is the following:
\begin{equation}
P(X_{(1)}>s | t< X_{(k)}\leq t+\Delta t) = \frac{P((X_{(1)}>s) \cap (t<
X_{(k)}\leq t+\Delta t))}{P(t<X_{(k)}\leq t+\Delta t)}
\end{equation}
Then we let $\Delta t \rightarrow 0$. Note that $k>1$.

Numerator: 

\setlength{\unitlength}{3947sp}%
%
\begingroup\makeatletter\ifx\SetFigFont\undefined%
\gdef\SetFigFont#1#2#3#4#5{%
  \reset@font\fontsize{#1}{#2pt}%
  \fontfamily{#3}\fontseries{#4}\fontshape{#5}%
  \selectfont}%
\fi\endgroup%
\begin{picture}(2566,435)(1118,-1711)
\thinlines
\put(1163,-1486){\circle{74}}
\put(3638,-1486){\circle{76}}
\put(1913,-1486){\circle{76}}
\put(2513,-1486){\circle{76}}
\put(3038,-1486){\circle{76}}
\put(1201,-1486){\line( 1, 0){2400}}
\put(3601,-1711){\makebox(0,0)[lb]{\smash{\SetFigFont{12}{14.4}{\rmdefault}{\mddefault}{\updefault}a}}}
\put(1876,-1711){\makebox(0,0)[lb]{\smash{\SetFigFont{12}{14.4}{\rmdefault}{\mddefault}{\updefault}s}}}
\put(2476,-1711){\makebox(0,0)[lb]{\smash{\SetFigFont{12}{14.4}{\rmdefault}{\mddefault}{\updefault}t}}}
\put(2926,-1711){\makebox(0,0)[lb]{\smash{\SetFigFont{12}{14.4}{\rmdefault}{\mddefault}{\updefault}t+dt}}}
\put(3151,-1411){\makebox(0,0)[lb]{\smash{\SetFigFont{12}{14.4}{\rmdefault}{\mddefault}{\updefault}n-k}}}
\put(2701,-1411){\makebox(0,0)[lb]{\smash{\SetFigFont{12}{14.4}{\rmdefault}{\mddefault}{\updefault}1}}}
\put(2101,-1411){\makebox(0,0)[lb]{\smash{\SetFigFont{12}{14.4}{\rmdefault}{\mddefault}{\updefault}k-1}}}
\put(1501,-1411){\makebox(0,0)[lb]{\smash{\SetFigFont{12}{14.4}{\rmdefault}{\mddefault}{\updefault}0}}}
\put(1126,-1711){\makebox(0,0)[lb]{\smash{\SetFigFont{12}{14.4}{\rmdefault}{\mddefault}{\updefault}0}}}
\end{picture}
\begin{equation}
P((X_{(1)}>s) \cap (t<X_{(k)}\leq t+\Delta t)) = {{n}\choose{0,
k-1, 1, n-k}} (\frac{t-s}{a})^{k-1} \cdot \frac{\Delta t}{a} \cdot
\frac{(a-t-\Delta t)^{n-k}}{a^{n-k}} +{\cal O} (\Delta t^2)
\end{equation}

Denominator:

\setlength{\unitlength}{3947sp}%
%
\begingroup\makeatletter\ifx\SetFigFont\undefined%
\gdef\SetFigFont#1#2#3#4#5{%
  \reset@font\fontsize{#1}{#2pt}%
  \fontfamily{#3}\fontseries{#4}\fontshape{#5}%
  \selectfont}%
\fi\endgroup%
\begin{picture}(2566,435)(1118,-1711)
\thinlines
\put(1163,-1486){\circle{74}}
\put(3638,-1486){\circle{76}}
\put(2851,-1486){\circle{76}}
\put(2026,-1486){\circle{76}}
\put(1201,-1486){\line( 1, 0){2400}}
\put(3601,-1711){\makebox(0,0)[lb]{\smash{\SetFigFont{12}{14.4}{\rmdefault}{\mddefault}{\updefault}a}}}
\put(3151,-1411){\makebox(0,0)[lb]{\smash{\SetFigFont{12}{14.4}{\rmdefault}{\mddefault}{\updefault}n-k}}}
\put(1126,-1711){\makebox(0,0)[lb]{\smash{\SetFigFont{12}{14.4}{\rmdefault}{\mddefault}{\updefault}0}}}
\put(1501,-1411){\makebox(0,0)[lb]{\smash{\SetFigFont{12}{14.4}{\rmdefault}{\mddefault}{\updefault}k-1}}}
\put(2401,-1411){\makebox(0,0)[lb]{\smash{\SetFigFont{12}{14.4}{\rmdefault}{\mddefault}{\updefault}1}}}
\put(2701,-1711){\makebox(0,0)[lb]{\smash{\SetFigFont{12}{14.4}{\rmdefault}{\mddefault}{\updefault}t+dt}}}
\put(2026,-1711){\makebox(0,0)[lb]{\smash{\SetFigFont{12}{14.4}{\rmdefault}{\mddefault}{\updefault}t}}}
\end{picture}
\begin{equation}
P(t<X_{(k)}\leq t+\Delta t)= {{n}\choose{k-1, 1, n-k}}
(\frac{t}{a})^{k-1} \cdot \frac{\Delta t}{a} \cdot
(\frac{(a-t-\Delta t)}{a})^{n-k} +{\cal O} (\Delta t^2)
\end{equation}

Then 
\begin{eqnarray*}
P(X_{(1)}>s | t< X_{(k)}\leq t+\Delta t)&=& \frac{(t-s)^{k-1}
+{\cal O}
(\Delta t)}{t^{k-1}} +{\cal O} (\Delta t)\\ 
P(X_{(1)} >s | X_{(k)}=t)&=&\lim_{\Delta t \rightarrow 0}P(X_{(1)}>s | t<
X_(k)\leq t+\Delta t)= (\frac{t-s}{t})^{k-1}
\end{eqnarray*}

Now lets do this with feeling. 
Left for VLAD to write with a lot of feeling :-)
I did not understand a word that Rota said at this point....

\setlength{\unitlength}{3947sp}%
%
\begingroup\makeatletter\ifx\SetFigFont\undefined%
\gdef\SetFigFont#1#2#3#4#5{%
  \reset@font\fontsize{#1}{#2pt}%
  \fontfamily{#3}\fontseries{#4}\fontshape{#5}%
  \selectfont}%
\fi\endgroup%
\begin{picture}(2566,270)(1118,-1711)
\thinlines
\put(1163,-1486){\circle{74}}
\put(3638,-1486){\circle{76}}
\put(2251,-1486){\circle{76}}
\put(1201,-1486){\line( 1, 0){2400}}
\put(3601,-1711){\makebox(0,0)[lb]{\smash{\SetFigFont{12}{14.4}{\rmdefault}{\mddefault}{\updefault}a}}}
\put(1126,-1711){\makebox(0,0)[lb]{\smash{\SetFigFont{12}{14.4}{\rmdefault}{\mddefault}{\updefault}0}}}
\put(2026,-1711){\makebox(0,0)[lb]{\smash{\SetFigFont{12}{14.4}{\rmdefault}{\mddefault}{\updefault}X\_\{k\}=t}}}
\end{picture}


\begin{equation}
P(A|Y=t)= \lim_{\Delta t \rightarrow 0} \frac{P(A \cap t < Y\leq t+
\Delta t)}{P(t< Y \leq t+ \Delta t)}
\end{equation}

We want to discuss the continuous form of the most important law in
probability which is the law of alternatives.

\noindent {\bf Continuous Analog of Law of Alternatives}:

Recall: $\Pi =$ partition of $\Omega$.
\begin{eqnarray*}
P(A)&=& \sum_{B\epsilon \Pi} P(A|B)P(B) \\
P(A|Y=t) &\propto& P(A | t < Y \leq t+ \Delta t)
\end{eqnarray*}

Take the whole line and split it into intervals which will later will
be made little.
 
\setlength{\unitlength}{3947sp}%
%
\begingroup\makeatletter\ifx\SetFigFont\undefined%
\gdef\SetFigFont#1#2#3#4#5{%
  \reset@font\fontsize{#1}{#2pt}%
  \fontfamily{#3}\fontseries{#4}\fontshape{#5}%
  \selectfont}%
\fi\endgroup%
\begin{picture}(3474,270)(664,-1711)
\thinlines
\put(2101,-1486){\circle{76}}
\put(976,-1486){\circle{74}}
\put(1576,-1486){\circle{76}}
\put(3826,-1486){\circle{76}}
\put(3226,-1486){\circle{76}}
\put(2701,-1486){\circle{76}}
\put(676,-1486){\line( 1, 0){3450}}
\put(1351,-1711){\makebox(0,0)[lb]{\smash{\SetFigFont{12}{14.4}{\rmdefault}{\mddefault}{\updefault}t\_-1}}}
\put(1951,-1711){\makebox(0,0)[lb]{\smash{\SetFigFont{12}{14.4}{\rmdefault}{\mddefault}{\updefault}t\_0}}}
\put(2551,-1711){\makebox(0,0)[lb]{\smash{\SetFigFont{12}{14.4}{\rmdefault}{\mddefault}{\updefault}t\_1}}}
\put(3076,-1711){\makebox(0,0)[lb]{\smash{\SetFigFont{12}{14.4}{\rmdefault}{\mddefault}{\updefault}t\_2}}}
\put(3676,-1711){\makebox(0,0)[lb]{\smash{\SetFigFont{12}{14.4}{\rmdefault}{\mddefault}{\updefault}t\_3}}}
\end{picture}
\begin{eqnarray*}
t_{i+1} &=& t_i + \Delta t_i \\
(t_i < Y \leq t_{i+1}) &=& \{w: t_i < Y(w) \leq t_{i+1}\} \subseteq \Omega
\end{eqnarray*}
Events $(t_i < Y \leq t_{i+1})$ are a partition of the sample
space. Therefore we can apply the classical law of alternatives as
previously obtained. 
\begin{eqnarray*}
P(A) &=& \sum_{i} P(A | t_i < Y \leq t_{i+1})P(t_i < Y \leq t_{i+1}) \\
dens(Y=t)&=&f(t) \\
P(t_i < Y \leq t_{i+1}) &=& \int_{t_i}^{t_{i+1}} f(s)ds
\end{eqnarray*}

By the mean value theorem of integral calculus we have
\begin{equation}
P(T_i < Y \leq t_{i+1}) = (t_{i+1}-t_{i})f(\bar{t_i})
\end{equation}
Therefore, 
\begin{equation}
P(A) = \sum_{i} P(A | t_i < Y \leq t_{i+1}) (t_{i+1}-t_i)f(\bar{t_i})
\end{equation}
In the limit, as all $t_{i+1}-t_i \rightarrow 0$ this becomes
\begin{equation}
P(A)= \int_{-\infty}^{\infty} P(A|Y=t)f(t)dt
\end{equation}

\noindent {\bf Example: Gaps in the Dirichlet Process}:
\begin{eqnarray*}
P(L_2 >t) &=& \int_{s=0}^{a-t} P(L_2>t| L_1 =s) \cdot dens(L_1=s)ds \\
dens(L_1=t) &=& dens(X_{(1)}=t) = \tder{}{t} P(X_1 \leq t) = \tder{}{t}(1-
\frac{(a-t)^n}{a^n}) \\
dens(L_1 =t) &=& \frac{n (a-t)^{n-1}}{a^n} \\
dens(L_2>t | L_1=s) &=& \frac{(n-a)(a-s-t)^{n-2}}{(a-s)^{n-1}} \\
P(L_2>t | L_1=s) &=& \frac{(a-s-t)^{n-1}}{(a-s)^{n-1}} \\
P(L_2 \leq t) &=& \int P(L_2 \leq t | L_1 =s) \cdot dens(L_1 =s) ds \\
dens(L_2=t) &=& \int dens(L_2=t | L_1=s) \cdot dens(L_1) ds \\
P(L_2>t) &=& \int_{s=0}^{a-t_1} \frac{(a-s-t)^{n-1}}{(a-s)^{n-1}} \cdot
n \cdot \frac{(a-s)^{n-1}}{a^n} ds \\
&=& n \cdot \int_{s=0}^{a-t_1} \frac{(a-s-t)^{n-1}}{a^n} ds =
-\left. \frac{(a-s-t)^n}{a^n}\right|^{s=0}_{s=a-t}\\
&=& \frac{(a-t)^n}{a^n}
\end{eqnarray*}
So we have a proven that the gaps are identically distributed as we
previously has assumed.









{\Large 18.313 Lecture 1, Wednesday, February 3, 1999}\newline
{\large Lecture by Prof. G.-C. Rota}\newline
Transcribed by Wei-An Yu\\

\section{Famous and Unsolved Problems}

\subsection{1.  Self-Avoiding Random Walk}

You start on the origin, $(0, 0)$, of a 2-dimensional rectangular coordinate system.  Toss a 4 sided coin where all 4 sides have equal probability of showing and each of the 4 sides corresponds to a direction (either up/down/right/left or north/south/east/west).  The probability of ever getting back to the origin is 1.  The 3-dimensional analogue is similar with a 6 sided coin corresponding to up/down/left/right/forward/backward in a 3 dimensional rectangular coordinate system.  In the 3-dimensional case, the probability of ever returning to the origin is approximately 0.3278.

Imagine a circular (or spherical in the 3-dimensional case) barrier of radius $r$ centered at the origin.  What is the probability of hitting the barrier in terms of $r$ and $n$ steps?  This problem is unsolved.

What is the probability that in the first $n$ steps, the random walk will never return to a point it has visited?  This problem is unsolved.  We do know that for $p_m$, the probability of a self-avoiding $n$step random walk, there exists $\lim_{n \rightarrow \infty} \sqrt[n]{p_m}$.

\subsection{2.  Pennies on a Carpet}

You have $n$ identical pennies and you drop them on a carpet (within the carpet boundary).  What is the probability that no 2 pennies overlap?  The solution is unknown.  In statistical mechanics, this relates to the hard spheres problem.

\subsection{3.  Cell Growth}

Start with a single cell represented by a square.  Toss a 4 sided die with each side corresponding to the growth of a new cell (square).  Repeat this process $n$ times for the newly formed cell.  The 3-dimensional analogue replaces the square with a cube and uses a 6 sided die.  What is the general shape?  Do islands form?

\subsection{4.  Cluster Analysis}

Imagine you have a large number of points of data plotted.  How do you analyze the data, which may be multi-dimensional, into categories?  How could this be used in anomaly detection?  There are no direct methods of solving this.  Current methods include solving indirectly, inventing theories, etc.

\section{The Grammar of Probability}

We have the sample space, $\Omega$, whose elements, $\omega$, are called sample points.  A family, $\varepsilon$, of subsets of $\Omega$, is called an event, such that:

\begin{enumerate}
\item null set, $\emptyset$, is an event
\item if $A$ is an event, then its complement, $\Omega - A = A^c$, is also an event
\item if $A$ and $B$ are events, then $A \cap B$, the intersection of $A$ and $B$, is an event
\item if $A_1, A_2, A_3, ...$ is a finite or infinite sequence of disjoint events, then $A_1 \cup A_2 \cup A_3 \cup ...$, the union of $A_1, A_2, A_3, ...$ is an event
\end{enumerate}

Example 1.  if $\Omega$ is a finite set, every subset of $\Omega$ is an event\\
Example 2.  The Bernoulli Process

\begin{eqnarray*}
\Omega &=& \{ \omega = (\omega_1, \omega_2, \omega_3, ...)\}, \; \omega_n = 0 \; or \; 1\\
\varepsilon :\: H_n &=& event\:that\:nth\:toss\:is\:head\\
&=& the\:set\:of\:all\:sample\:points\:\omega = (\omega_1, \omega_2, ..., \omega_n, \omega_{n + 1}, ...)\\
\end{eqnarray*}

For which $\omega_n = 1$.

Any subset of $\Omega$ obtainable from any of the $H_n$ by successive aplication of 1, 2, 3, and 4 is an event.
